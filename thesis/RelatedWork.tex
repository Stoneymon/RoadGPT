\chapter{Related Work}
In this chapter, related works will be discussed.

\section{Generation of Virtual Roads}
Gambi et al. point out that the conventional methods of testing autonomous vehicles in actual traffic situations pose significant risks, are expensive, and have resulted in fatal incidents. In response to these challenges, the authors devised a solution known as AsFault. This automated tool tests software for self-driving cars by creating virtual road networks and simulating driving scenarios. It utilizes a combination of a genetic algorithm and procedural content generation to create road networks that expose potential safety issues. The primary objective is identifying bugs and problematic environmental conditions that can cause self-driving cars to fail. AsFault focuses explicitly on testing the software's lane-keeping capability. By evolving the road networks based on a fitness function that considers the distance between the car and the center of the lane, AsFault generates test cases where the car deviates from its intended path. The tool can be installed as a Python package and relies on the BeamNG.research driving simulator. Test cases generated by AsFault are saved in JSON format and can be re-executed. It also provides visualizations of the results, including the number of instances where the car goes out of bounds and the coverage of road segments achieved by the generated test suite \cite{GambiMF}.

In their research, Tang et al. showed that High-definition (HD) maps, which are critical for evaluating ADSs in simulated environments, come with certain limitations. They include features such as road networks, traffic lights, and signs, which essentially influence the range of testing scenarios, directly affecting the tests' effectiveness. However, several issues arise with these HD maps. Firstly, commercial HD maps often have a significant quantity of repeated elements, such as junctions of similar shapes, leading to a lack of diversity in the testing scenarios. Secondly, the HD maps included with open-source simulators are usually relatively small, possessing a restricted number of roads and junctions, further limiting the diversity of testing scenarios. Thirdly, the unique junction structures found in different cities or regions make it time-consuming to test ADSs thoroughly across various HD maps from different locations. Lastly, the process of manually creating diverse HD maps using specific tools is quite demanding and labor-intensive, especially considering the myriad of scenario requirements \cite{TangZYZRLZC}.

To address these issues, Tang et al. introduced an innovative solution called Feat2Map. This is an automated framework for generating HD maps based on essential features, tailored for simulation testing of autonomous driving systems. It operates by taking existing maps, extracting crucial features, and then utilizing combinatorial sampling to generate grid-layout HD maps. Feat2Map is also designed to accept manual feature configurations, providing a degree of customization flexibility. The goal of this framework is to eradicate duplicated intersections, while also ensuring a maximum diversity of intersection structures and potential scenarios. The end result is a concise HD map that maintains a comparable level of scenario diversity \cite{TangZYZRLZC}.

The Search-Based and Fuzz Testing (SBFT) workshop, previously focused on Search-Based Software Testing (SBST), aims to bring together researchers and practitioners from SBST, Fuzzing, and Software Engineering to discuss and advance the automation of software testing. The workshop aims to explore the application of search and fuzzing techniques in testing, as well as their integration with other software engineering areas. The workshop includes a research track, keynotes, testing tool competitions, and a panel discussion to facilitate knowledge sharing and foster new advancements in SBFT research \cite{SBFT}.

One of the competitions inside the SBFT workshop is the CPS (Cyber-Physical Systems) Testing Tool Competition, which aims to encourage researchers to investigate the problem of testing safety-critical CPSs (such as self-driving cars) and provide a shared framework for benchmarking test generators. The competition involves generating the highest number of diverse failure-inducing inputs, i.e., valid virtual roads that cause the autonomous vehicle to drive out of the lane. The infrastructure detects a failure each time the ego-car (partially) drives outside the lane, and the competitors are ranked based on various aspects of test generation, including their ability to generate valid test cases and trigger out-of-bounds episodes \cite{GambiJRZ}.

Pathrudkar et al. introduced the SceVar (Scenario Variations) database to enhance the realism of simulations. This database improves autonomous vehicle testing by generating realistic variations of scenarios based on real-world driving data. These variations enable a wide range of scenarios for simulation-based regression testing. The SceVar database significantly advances autonomous vehicle testing by incorporating key features such as semantic data models, real-world statistical analysis, statistical insights, and smart scenario variations. Leveraging semantic data models, it comprehensively represents static and dynamic entities in the road-traffic ecosystem, fostering intuitive interactions. Through the analysis of real-world driving data, the database extracts authentic traffic patterns, enabling the creation of numerous realistic scenario variations for simulation-based testing. The integration of statistical analysis allows for insights, pattern identification, and a focused reduction of the simulation state space. Moreover, the database facilitates the creation of smart scenario variations based on statistical insights, covering a wide range of test cases while minimizing the number of simulation runs. Collectively, these features contribute to the efficiency, accuracy, and cost-effectiveness of autonomous vehicle testing in simulation-based environments \cite{PathrudkarMSC}.