\chapter{Summary}

The objective of this thesis was to create a tool utilizing generative AI to generate virtual roads for testing self-driving vehicles. The system was designed to streamline the process of testing autonomous vehicles by converting textual road descriptions into coordinate-based nodes. These nodes are used to construct the road within the simulation environment. This approach eliminates the need to write complicated algorithms.

The tool's implementation involved selecting appropriate technologies, such as the programming language, the chosen LLM, and the simulator. The evaluation comprised a quantitative analysis of the number of generated roads and instances where the car deviated from its lane, alongside a survey gathering feedback from target users regarding the generated roads.

Although the model didn't effectively stress the lane-keeping functionality, it's important to note that it wasn't trained for that purpose. Nevertheless, the model generally produced roads that aligned with the provided descriptions.

In conclusion, the tool developed in this thesis presents a promising solution for generating virtual roads using generative AI. The study results suggest that the tool holds the potential to serve as a foundation for future research in this field.


\section{Future work}

While this study established the groundwork for using generative AI in road generation, there are still many areas to explore to enhance the performance of AI models in this task.

\textbf{Fine-Tuning a model}: By compiling a dataset containing road descriptions paired with descriptive prompts as labels, it is possible to fine-tune a model specifically for this purpose and enhancing its performance. This dataset can be constructed either by labeling roads generated by an existing tool or by manually generating roads based on provided prompts.

\textbf{Training a new model}: Training a new model allows for fine-tuning it for specific purposes, ensuring adherence to all relevant rules. This can be achieved by training the model on either artificially generated roads or real roads. While training on generated roads can effectively stress test the lane-keeping functionality of the self-driving agent, training on real-world roads ensures that the model closely mimics real-world scenarios.

\textbf{Terrain modification}: At the time of writing, BeamNgpy introduced new functionality allowing users to import heightmaps and enable terrain modification. This introduces a challenge as road sections might be obscured by terrain. Moreover, when combined with traffic, it not only tests the car's lane-keeping capabilities but also its ability to avoid collisions. 

\textbf{Road Networks}: Expanding the functionality of road generation tools to encompass entire road networks, including features such as junctions, offers numerous advantages. It provides a more comprehensive evaluation of autonomous vehicles in varied driving environments, testing their ability to navigate complex intersections and merge lanes. Including diverse lane markings enhances realism, resulting in more precise performance evaluations and better preparation for real-world deployment. In essence, shifting from individual roads to complete road networks enhances testing precision and helps in the development of safer autonomous driving systems.